%% The comment character in TeX / LaTeX is the percent character.
%% The following chunk is called the header

\documentclass{article}	% essential first line
\usepackage{float}		% this is to place figures where requested!
\usepackage{times}		% this uses fonts which will look nice in PDF format
\usepackage{graphicx}		% needed for the figures
\usepackage{url}

%% Here I adjust the margins

\oddsidemargin -0.25in		% Left margin is 1in + this value
\textwidth 6.75in		% Right margin is not set explicitly
\topmargin 0in			% Top margin is 1in + this value
\textheight 9in			% Bottom margin is not set explicitly
\columnsep 0.25in		% separation between columns

%% Define a macro for inserting postscript images
%% ==============================================
%% This is a macro which nominally takes 3 parameters, 
%% it would be used as follows to insert and encapsulated postscript
%% image at the location where it is used.
%%
%% \EPSFIG{epsfilename}{caption}{label}
%% - epsfilename is the name of the encapsulated postscript file to be
%%               inserted at this location
%% - caption is the text to be shown as the figure caption, it will be
%%           prepended by Figure X.  The number X can be referenced
%%           using the label parameter.
%% - label is a name given to the figure, it can be referenced using the
%%         \ref{label} command.

\def\EPSFIG[#1]#2#3#4{		% Don't be scared by this monsrosity
\begin{figure}[H]		% it is a macro to save typing later
\begin{center}			% 
\includegraphics[#1]{#2}	%
\end{center}			%
\caption{#3}			%
\label{#4}			%
\end{figure}			%
}				%


%% Define the fields to be displayed by a \maketitle command

\author{First Author, Second Author}
\title{A Template For Writing Lab Reports in \LaTeX}

%%
%% Header now finished
%%

\begin{document}		% Critical
\twocolumn
\thispagestyle{empty}		% Inhibit the page number on this page
\maketitle			% Use the \author, \title and \date info

%% Next comes the abstract, notice the curly-braces surrounding the
%% text.

\abstract{A document was prepared which gives a framework which can be
used to write documents suitable for lab reports.}

\section{Introduction}
\LaTeX\ is a typesetting language\cite{LAMPORT}.  It is actually a set
of \textit{macros} for \TeX, which is a typesetting system.  \TeX\ was
written by Donald Knuth, a celebrated computer scientist and
mathematician, in reaction to his disatisfation with available
software.  Much as Unix is an operating system which appeals to people
who write programs, \TeX\ and \LaTeX\ tend to appeal to technically
minded people.

This document is written to expose you to some of the features of
\LaTeX\ which might make it suitable for preparing lab reports.  It
does not cover how to compile documents, this is covered on the
\textit{Wiki} page for this template\cite{WIKI-LATEX-TEMPLATE}.

This template is not provided as a part of any exercise, you do not
have to use it, it is provided as a resource which you may use if you
are interested.

\section{Typesetting}
The following is quoted from Wikipedia\cite{WIKIPEDIA-LATEX}:
\begin{quote}
\LaTeX\ is based on the idea that authors should be able to concentrate
on writing within the logical structure of their document, rather than
spending their time on the details of formatting. It encourages the
separation of formatting from content, whilst still allowing manual
typesetting adjustments where needed. By keeping the formatting
details in a separate file from the text, it is often regarded as
superior to word processors and most other desktop publishing systems,
which allow trivially easy visual layout changes but tend to
intertwine content and form so tightly that consistency and automation
are often difficult.
\end{quote}

\section{Citations}
Notice how in the Introduction section I made reference to an item in
the References section at the end of the document.  In the
bibliography file \texttt{report.bib} the information about this
reference is stored and labeled \texttt{LAMPORT}.  Anywhere I wish to
cite this reference it is just a matter of entering the command
\texttt{$\backslash$cite\{LAMPORT\}}.  \LaTeX\ will keep track of the
numbers, you just have to use the labels.

\section{Using Equations}
Sometimes in a lab report you might want to insert the odd equation,
it is important to refer to these by number.  Again we can let
\LaTeX\ keep track of the numbers and use labels.  It is not at all
required, but an unspoken convention that most people use upcase for
labels.  Labels can contain hyphens (-), but not underscores (\_).

\begin{equation}
\mu = \frac{1}{N}\sum_{i=1}^N x_i.
\label{EQ-DEFMEAN}
\end{equation}
%It is important that no blank line appear here as that starts a new paragraph.
Equation \ref{EQ-DEFMEAN} shows a common definition for the mean,
$\mu$, where $x_i$ is the value of the $i$th sample, $N$ is the number of samples.
First note that, as required, all symbols have been defined.
Also note that the equation has a period. Equations are part of the text and
should follow punctuation rules. That is they should have periods, commas
etc as required. A good rule of thumb is to treat the equation like
a sentence with the equal sign as the verb. Thus the above formula
reads:
$\mu$ equals one over $N$ times the sum of $x_i$ from 1 to $N$. As you can
see, this sentence
definitely requires a period.

There are two modes in \LaTeX, \textit{paragraph mode} and \textit{math
mode}.  \LaTeX\ is in paragraph mode when it is processing ordinary
prose\cite{DILLER}.  Between the
\texttt{$\backslash$begin\{equation\}} and
\texttt{$\backslash$end\{equation\}} commands \LaTeX\ is in math mode,
this allows access to a vast array of mathematical symbols.  You can
switch briefly into math mode using the dollar character (\$) as a
delimiter.

\section{Figures}
Arbitrary encapsulated-postscript can be displayed in \LaTeX\ using
the figure environment.  I always use the ``H'' position specifier in
conjunction with the \texttt{$\backslash$usepackage\{float\}} command in the
header of the document.

\begin{figure}[H]
\begin{center}
\includegraphics[scale=0.65]{pendulum.eps}
\end{center}
\caption{Variation of period of a pendulum as a function of length.}
\label{FIG-PEND}
\end{figure}

Please note that the graph shown in figure~\ref{FIG-PEND} is only to
show how to insert a graph, not in any way how to format a graph.  It
is a graph which I made to reproduce an image in the Phys-257/258 lab
notes a couple of years ago when they were being revised.
Unfortunately, as an example of a graph, it is repugnant.  On brief
inspection the following points come to mind:
\begin{itemize}
\item Where are the error bars?
\item A more descriptive x-axis caption would have been ``Pendulum length (m).''
\item Despite space restrictions, a better y-axis caption would have
  been ``Period (s).''
\item The numbering is on the small side, almost unreadable.
\end{itemize}

One of the nice features of \LaTeX\ is that one can define a macro to
cut down on the amount of redundant typing required.  This was the
purpose of the \texttt{EPSFIG} macro defined in the header of this
document.  Notice how the figure~\ref{FIG-GAUSS} is inserted using the
macro.

\EPSFIG[scale=0.5]{gaussian.eps}{A decorated gaussian}{FIG-GAUSS}

The Encapsulated Postscript image shown in figure~\ref{FIG-GAUSS} was
created using Gnuplot.

\section{Conclusions}
Hopefully this might be slightly helpful.  I urge you to look at some
other reference materials, such as Oetiker's paper\cite{OETIKER-NSSITL}.

%% This bit generates the references.  This part starts to get
%% slightly tricky.

\bibliographystyle{unsrt}	% Order by citation
\bibliography{report}

\end{document}
